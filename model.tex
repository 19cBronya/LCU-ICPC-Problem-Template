\documentclass[11pt]{ctexart}
\usepackage{amsmath, amssymb}
\usepackage{hyperref}
\usepackage{graphicx}
\usepackage{fancyhdr}
\usepackage{geometry}
\usepackage{array}
\usepackage{titlesec} 
\geometry{a4paper, margin=1in}

% 设置 \section* 的格式
\titleformat{\section}[hang]{\normalfont\Large\bfseries}{Problem \thesection.}{1em}{}[]

% 页眉页脚设置
\pagestyle{fancy}
\fancyhf{}
\fancyhead[L]{2024 聊城大学大学生程序设计竞赛 }
\fancyhead[R]{正式赛}
\fancyfoot[C]{\thepage}

\begin{document}



\newpage
% 问题 A 的描述
\section*{Problem A.A题标题}

A题题面

\textbf{输入}

输入描述

\textbf{输出}

输出描述

\textbf{样例}

\begin{table}[h!]
\centering
\begin{tabular}{|p{7cm}|p{7cm}|}
\hline
\textbf{输入样例} & \textbf{输出样例} \\
\hline
\begin{minipage}[t]{7cm}
\begin{verbatim}

输入样例放这里


\end{verbatim}
\end{minipage} &
\begin{minipage}[t]{7cm}
\begin{verbatim}

输出样例放这里

\end{verbatim}
\end{minipage} \\
\hline
\end{tabular}
\end{table}

\textbf{样例解释}

这里是样例解释

% \textbf{备注}

% 此处为题目的备注内容,例如题目的特殊说明或提示等。

% 后续题目可以按照同样的格式添加
% \newpage

\end{document}
